\documentclass[a4paper, 12pt, oneside]{article}
\usepackage[T1]{fontenc}
\usepackage{aurical}
\usepackage{booktabs}
\usepackage{textalpha}
\usepackage{url}
\setlength{\emergencystretch}{15pt}
\usepackage{fancyhdr}
\usepackage{amssymb}
\usepackage{array}
\usepackage{imakeidx}
\usepackage{qtree}
\usepackage{microtype}
\usepackage{sectsty}
\usepackage[titles]{tocloft}

\allsectionsfont{\Fontauri}
\sectionfont{\Fontauri\Huge}
\subsectionfont{\Fontauri\LARGE}
\subsubsectionfont{\Fontauri\Large}

\begin{document}
\Fontauri
\renewcommand{\contentsname}{
\Fontauri{Index}
}

\renewcommand{\cftsecfont}{\Fontauri}
\renewcommand{\cftsubsecfont}{\Fontauri}
\renewcommand{\cftsubsubsecfont}{\Fontauri}

% fix toc page numbers
\let\origcftsecfont\cft
\let\origcftsecpagefont\cftsecpagefont
\let\origcftsecafterpnum\cftsecafterpnum
\renewcommand{\cftsecpagefont}{\Fontauri{\origcftsecpagefont}}
\renewcommand{\cftsecafterpnum}{\Fontauri{\origcftsecafterpnum}}
\let\origcftsubsecpagefont\cftsubsecpagefont
\let\origcftsubsecafterpnum\cftsubsecafterpnum
\renewcommand{\cftsubsecpagefont}{\Fontauri{\origcftsubsecpagefont}}
\renewcommand{\cftsubsecafterpnum}{\Fontauri{\origcftsubsecafterpnum}}
\let\origcftsubsubsecpagefont\cftsubsubsecpagefont
\let\origcftsubsubsecafterpnum\cftsubsubsecafterpnum
\renewcommand{\cftsubsubsecpagefont}{\Fontauri{\origcftsubsubsecpagefont}}
\renewcommand{\cftsubsubsecafterpnum}{\Fontauri{\origcftsubsubsecafterpnum}}

\renewcommand\thefootnote{\Fontauri{\arabic{footnote}}}

\begin{titlepage} % Suppresses headers and footers on the title page
	\centering % Centre everything on the title page
	\scshape % Use small caps for all text on the title page

	%------------------------------------------------
	%	Title
	%------------------------------------------------
	
	\rule{\textwidth}{1.6pt}\vspace*{-\baselineskip}\vspace*{2pt} % Thick horizontal rule
	\rule{\textwidth}{0.4pt} % Thin horizontal rule
	
	\vspace{0.75\baselineskip} % Whitespace above the title

        {\Huge On Meteorites \\} % Title
	
	\vspace{0.75\baselineskip} % Whitespace below the title
	
	\rule{\textwidth}{0.4pt}\vspace*{-\baselineskip}\vspace{3.2pt} % Thin horizontal rule
	\rule{\textwidth}{1.6pt} % Thick horizontal rule
	
	\vspace{1\baselineskip} % Whitespace after the title block
	
	%------------------------------------------------
	%	Subtitle
	%------------------------------------------------
	
	{By \scshape\Large Henry Clifton Sorby, FRS, FGS, \emph{etc.}\\} % Subtitle or further description
	
	\vspace*{1\baselineskip} % Whitespace under the subtitle
	
	%------------------------------------------------
	%	Editor(s)
	%------------------------------------------------

 	{Edited By \scshape Solar Anamnesis\\} % Subtitle or further description

	\vspace{1\baselineskip} % Whitespace before the editors

    %------------------------------------------------
	%	Cover photo
	%------------------------------------------------
	
	%\includegraphics[scale=1]{cover}
	
	%------------------------------------------------
	%	Publisher
	%------------------------------------------------
		
	\vspace*{\fill}% Whitespace under the publisher logo
	
	1864 --- 1877.% Publication year
	
	{\small Proceedings of the Royal Society. \\ Quarterly journal of the Geological Society. \\ Nature. } % Publisher

	\vspace{1\baselineskip} % Whitespace under the publisher logo

    {\small Solar Anamnesis Edition} % Publication year
	
	{\small CC0 1.0 Universal } % Publisher
\end{titlepage}
\setlength{\parskip}{1mm plus1mm minus1mm}
\setcounter{tocdepth}{3}
\setcounter{secnumdepth}{3}
\pagestyle{fancy}
\fancyhf{}
\cfoot{\Fontauri{\thepage}}
\tableofcontents
\clearpage
\Large
\section{On the Microscopical Structure of Meteorites.
}
\begin{center}
Received June 7, 1864.
\end{center}
\paragraph{}
For some time past I have endeavoured to apply to the study of meteorites the principles I have made use of in the investigation of terrestrial rocks, as described in my various papers, and especially in that on the microscopical structure of crystals (Quart. Journ. Geol. Soc. 1858, vol. 14. p. 453). I therein showed that the presence in crystals of "fluid-, glass-, stone-, or gas-cavities" enables us to determine in a very satisfactory manner under what conditions the crystals were formed. There are also other methods of inquiry still requiring much investigation, and a number of experiments must be made which will occupy much time; yet, not wishing to postpone the publication of certain facts, I purpose now to give a short account of them, to be extended and completed on a subsequent occasion.\footnote{\Fontauri{The names given thus (Stannern) indicate what meteorites I more particularly refer to in proof of the various facts previously stated.}}

In the first place it is important to remark that the olivine of meteorites contains most excellent "glass-cavities," similar to those in the olivine of lavas, thus proving that the material was at one time in a state of igneous fusion. The olivine also contains "gas-cavities," like those so common in volcanic minerals, thus indicating the presence of some gas or vapour (Assun, Parnallee). To see these cavities distinctly, a carefully prepared thin section and a magnifying power of several hundred are required. The vitreous substance found in the cavities is also met with outside and amongst the crystals, in such a manner as to show that it is the uncrystalline residue of the material in which they were formed (Mezö-Madaras, Parnallee). It is of a claret or brownish colour, and possesses the characteristic structure and optical properties of artificial glasses. Some isolated portions of meteorites have also a structure very similar to that of stony lavas, where the shape and mutual relations of the crystals to each other prove that they were formed \emph{in situ}, on solidification. Possibly some entire meteorites should be considered to possess this peculiarity (Stannern, New Concord), but the evidence is by no means conclusive, and what crystallization has taken place \emph{in situ} may have been a secondary result; whilst in others the constituent particles have all the characters of broken fragments (L'Aigle). This sometimes gives rise to a structure remarkably like that of consolidated volcanic ashes, so much, indeed, that I have specimens which, at first sight, might readily be mistaken for sections of meteorites. It would therefore appear that, after the material of the meteorites was melted, a considerable portion was broken up into small fragments, subsequently collected together, and more or less consolidated by mechanical and chemical actions, amongst which must be classed a segregation of iron, either in the metallic state or in combination with other substances. Apparently this breaking up occurred in some cases when the melted matter had become crystalline, but in others the forms of the particles lead me to conclude that it was broken up into detached globules whilst still melted (Mezö-Madaras, Parnallee). This seems to have been the origin of some of the round grains met with in meteorites; for they occasionally still contain a considerable amount of glass, and the crystals which have been formed in it are arranged in groups, radiating from one or more points on the external surface, in such a manner as to indicate that they were developed after the fragments had acquired their present spheroidal shape (Assun, \emph{etc.}). In this they differ most characteristically from the general type of concretionary globules found in terrestrial rocks, in which they radiate from the centre; the only case that I know at all analogous being that of certain oolitic grains in the Kelloways rock at Scarborough, which have undergone a secondary crystallization. These facts are all quite independent of the fused black crust.

Some of the minerals in meteorites, usually considered to be the same as those in volcanic rocks, have yet very characteristic differences in structure (Stannern), which I shall describe at greater length on a future occasion. I will then also give a full account of the microscopical structure of meteoric iron as compared with that produced by various artificial processes, showing that under certain conditions the latter may be obtained so as to resemble very closely some varieties of meteoric origin (Newstead, \emph{etc.}).

There are thus certain peculiarities in physical structure which connect meteorites with volcanic rocks, and at the same time others in which they differ most characteristically, --- facts which I think must be borne in mind, not only in forming a conclusion as to the origin of meteorites, but also in attempting to explain volcanic action in general. The discussion of such questions, however, should, I think, be deferred until a more complete account can be given of all the data on which these conclusions are founded.
\clearpage
\section{On the Conclusion to be deduced from the Physical Structure of some Meteorites.}
\begin{center}
1864.
\end{center}
\paragraph{}
The microscopical study of thin sections of meteorites had led the author to conclude that their earliest condition of which we have evidence was that of igneous fusion, as indicated by the crystals of olivine containing 'glass-cavities,' like those characteristic of the minerals in terrestrial volcanic rocks. (See Quart. Jour. Geol. Soc., vol. 14. p. 453; and Proceed. Roy. Soc., vol. 13. p. 333) There are, however, some meteorites, of which the 'Pallas Iron' may be taken as the type, consisting of a mixture of iron and olivine; and, if these were melted artificially, there can be no doubt, that, the iron being so much more dense would almost immediately sink to the bottom, and the olivine would rise to the top, like the slag in an iron-furnace. This at first sight appears to be strongly opposed to the supposition of igneous fusion; but the author contended that, since the force which would tend to separate the iron and olivine would vary with the force of gravitation, whilst the resistance to separation would be chiefly cohesion almost independent of it, if the fusion had taken place where the force of gravitation was very small, the iron and olivine might have remained fused and mixed together long enough to allow of slow crystallization. Hence he argued that such meteorites furnish us with physical evidence of having been formed where the force of gravitation was much smaller than on our globe, either near the surface of a very small planetary body, or towards the centre of a larger, which has since been broken into fragments.
\clearpage
\section{On the Physical History of Meteorites.}
\begin{center}
Broomfield, Sheffield: July 1865.
\end{center}
\paragraph{}
Though I am most willing to admit that much remains to be learned before we can look upon the following theory as anything more than provisional, yet at all events it serves to unite a great number of facts, and is not opposed to any with which I am now acquainted. I shall describe the facts and discuss the objections to this and other theories in a communication to the Royal Society.

As shown in my paper in the 'Proceedings of the Royal Society,' (13. 333), there is good proof of the material of meteorites having been to some extent fused, and in the state of minute detached particles. I had also met with facts which seemed to show that some portions had condensed from a state of vapour; and I expected that it would be requisite to adopt a modified nebular hypothesis, but hesitated until I had obtained more satisfactory evidence. The character of the constituent particles of meteorites and their general microscopical structure differ so much from what is seen in terrestrial volcanic rocks, that it appears to me extremely improbable that they were ever portions of the moon, or of a planet, which differed from a large meteorite in having been the seat of a more or less modified volcanic action. A most careful study of their microscopical structure leads me to conclude that their constituents were originally at such a high temperature that they were in a state of vapour, like that in which many now occur in the atmosphere of the sun, as proved by the black lines in the solar spectrum. On cooling, this vapour condensed into a sort of cometary cloud, formed of small crystals and minute drops of melted stony matter, which afterwards became more or less devitrified and crystalline. This cloud was in a state of great commotion, and the particles moving with great velocity were often broken by collision. After collecting together to form larger masses, heat, generated by mutual impact, or that existing in other parts of space through which they moved, gave rise to a variable amount of metamorphism. In some few cases, when the whole mass was fused, all evidence of a previous history has been obliterated; and on solidification a structure has been produced quite similar to that of terrestrial volcanic rocks. Such metamorphosed or fused masses were sometimes more or less completely broken up by violent collision, and the fragments again collected together and solidified. Whilst these changes were taking place, various metallic compounds of iron were so introduced as to indicate that they still existed in free space in the shape of vapour, and condensed amongst the previously formed particles of the meteorites. At all events, the relative amount of the metallic constituents appears to have increased with the lapse of time, and they often crystallized under conditions differing entirely from those which occurred when mixed metallic and stony materials were metamorphosed, or solidified from a state of igneous fusion in such small masses that the force of gravitation was too weak to separate the constituents, although they differ so much in specific gravity. (Report of Brit. Assoc. 1864) Possibly, however, some meteoric irons have been produced in this manner by the occurrence of such a separation. The hydrocarbons with which some few meteorites are impregnated, may have condensed from a state of vapour at a relatively late period.

I therefore conclude provisionally that meteorites are records of the existence in planetary space of physical conditions more or less similar to those now confined to the immediate neighbourhood of the sun, at a period indefinitely more remote than that of the occurrence of any of the facts revealed to us by the study of Geology --- at a period which might, in fact, be called \emph{pre-terrestrial}.
\clearpage
\section{On the Structure and Origin of Meteorites.}
\begin{center}
Abstract of lecture delivered by H. C. Sorby, RFS, \emph{etc.}, at the Museum, South Kensington, on March 10, 1877.
\end{center}
\paragraph{}
The study of meteorites is naturally divisible into several very distinct branches of inquiry. Thus in the first place we may regard them as shooting stars, and observe and discuss their radiant points and their relation to the solar system. This may be called the astronomical aspect of the question. Then, when solid masses fall to the ground, we may study their chemical composition as a whole, or that of the separate mineral constituents; and lastly, we may study their mechanical structure, and apply to this investigation the same methods which have yielded such important results in the case of terrestrial rocks. So much has been written on the astronomical, chemical, and mineralogical aspect of my subject by those far more competent than myself to deal with such questions, that I shall confine my remarks almost entirely to the mechanical structure of meteorites and meteoric irons, and more especially to my own observations, since they will, at all events, have the merit of greater originality and novelty. Time will, however, not permit me to enter into the detail even of this single department of my subject.

In treating this question it appeared to me very desirable to exhibit to you accurate reproductions of the natural objects, and I have therefore had prepared photographs of my original drawings, which we shall endeavour to show by means of the oxyhydrogen lime-light, and I shall modify my lecture to meet the requirements of the case, exhibiting and describing special examples, rather than attempt to give an account of meteorites in general. Moreover, since the time at my disposal is short, and their external characters may be studied to great advantage at the British Museum, I shall confine my remarks as much as possible to their minute internal structure, which can be seen only by examining properly prepared sections with more or less high magnifying powers.

By far the greater part of my observations were made about a dozen years ago. I prepared a number of sections of meteorites, meteoric irons, and other objects which might throw light on the subject, and my very best thanks are due to Prof. Maskelyne for having most kindly allowed me to thoroughly examine the very excellent series of thin sections, which had been prepared for him. During the last ten years my attention has been directed to very different subjects, and I have done little more than collect material for the further and more complete study of meteorites. When I have fully utilised this material I have no doubt that I shall be able to make the subject far more complete, and may find it necessary to modify some of my conclusions. I cannot but feel that very much more remains to be learned, and I should not have attempted to give an account of what I have so far done, if I had not been particularly asked to do so by Mr. Lockyer. At the same time I trust that I shall at all events succeed in showing that the microscopical method of study yields such well-marked and important facts, that in some cases the examination of only a single specimen serves to decide between rival theories.

In examining with the naked eye an entire or broken meteorite we see that the original external outline is very irregular, and that it is covered by a crust, usually, but not invariably black, comparatively thin, and quite unlike the main mass inside. This crust is usually dull, but sometimes, as in the Stannern meteorite, bright and shining, like a coating of black varnish. On examining with a microscope a thin section of the meteorite, cut perpendicular to this crust, we see that it is a true black glass filled with small bubbles, and that the contrast between it and the main mass of the meteorite is as complete as possible, and the junction between them sharply defined, except when portions have been injected a short distance between the crystals. We thus have a most complete proof of the conclusion that the black crust was due to the true igneous fusion of the surface under conditions which had little or no influence at a greater depth than 0.01 of an inch. In the case of meteorites of different chemical composition, the black crust has not retained a true glassy character, and is sometimes 0.02 of an inch in thickness, consisting of two very distinct layers, the internal showing particles of iron which have been neither melted nor oxidised and the oxide melted up with the surrounding stony matter. Taking everything into consideration, the microscopical structure of the crust agrees perfectly well with the explanation usually adopted, but rejected by some authors, that it was formed by the fusion of the external surface, and was due to the very rapid heating which takes place when a body moving with planetary velocity rushes into the earth's atmosphere --- a heating so rapid that the surface is melted before the heat has time to penetrate beyond a very short distance into the interior of the mass.

When we come to examine the structure of the original interior part of meteorites, as shown by fractured surfaces, we may often see with the naked eye that they are mottled in such a way as to have many of the characters of a brecciated rock, made up of fragments subsequently cemented together and consolidated. Mere rough fractures are, however, very misleading. A much more accurate opinion may be formed from the examination of a smooth flat surface. Facts thus observed led Reichenbach to conclude that meteorites had been formed by the collecting together of the fragments previously separated from one another in comets, and an examination of thin transparent sections with high magnifying powers and improved methods of illumination, proves still more conclusively their brecciated structure. The facts are, however, very complex, and some are not easily explained. Leaving this question for the present, I will endeavour to point out what appears to be the very earliest history of the material, as recorded by the internal structure.

It is now nearly twenty years since I first showed that the manner of formation of minerals and rocks may be learned from their microscopical structure. I showed that when crystals are formed by deposition from water or from a mass of melted rock, they often catch-up portions of this water or melted stone, which can now be seen as cavities containing fluid or glass. We may thus distinguish between crystalline minerals formed by purely aqueous or by purely igneous process; for example, between minerals in veins and minerals in volcanic lavas. In studying meteorites it appeared to me desirable, in the first place, to ascertain whether the crystalline minerals found in them were originally formed by deposition from water or from a melted stony material analogous to the slags of our furnace or the lava of volcanoes. One of the most common minerals in meteorites is olivine, and when met with in volcanic lavas this mineral usually contains only a few and small glass cavities in comparison with those seen in such minerals as augite. The crystals in meteorites are, moreover, only small, and thus the difficulty of the question is considerably increased. However, by careful examination with high magnifying power, I found well-marked glass-cavities, with perfectly fixed bubbles, the enclosed glass being sometimes of brown colour and having deposited crystals. On the contrary I have never been able to detect any trace of fluid-cavities, with moving bubbles, and therefore it is very probable, if not absolutely certain, that the crystalline minerals were chiefly formed by an igneous process, like those in lava, and analogous volcanic rocks. These researches require a magnifying power of 400 or 600 linear.

Passing from the structure of the individual crystals to that of the aggregate, we find that in some cases we have a structure in every respect analogous to that of erupted lavas, though even then there are very curious differences in detail. By methods like those adopted by Daubrée, there ought to be no more difficulty in artificially imitating the structure of such meteorites than in imitating that of our ordinary volcanic rocks. It is, however, doubtful whether meteorites of any considerable size uniformly possess this structure. The best examples I have seen are only fragments enclosed in the general mass of the Petersburg meteorite, which, like many others, has exactly the same kind of structure as that of consolidated volcanic tuff or ashes. This is well shown by the Bialystock meteorite, which is a mass of broken crystals and more complex fragments scattered promiscuously through a finer-grained consolidated dust-like ash.

Passing from this group of meteorites, which are more or less analogous to some of our terrestrial volcanic rocks, we must now consider the more common varieties, which are chiefly composed of olivine and other allied minerals. The Mezö-Madaras meteorite is an excellent illustration, since the outline of the fragments is well seen, on account of the surrounding consolidated fine material being of dark colour. In it we see more or less irregular spherical and very irregular fragments scattered promiscuously in a dark highly consolidated fine-grained base. By far the larger part of these particles do not either by their outline or internal structure furnish any positive information respecting the manner in which they were formed, but careful examination of this and other analogous meteorites, has enabled me to find that the form and structure of many of the grains is totally unlike that of any I have ever seen in terrestrial rocks, and points to very special physical conditions. Thus some are almost spherical drops of \emph{true glass} in the midst of which crystals have been formed, sometimes scattered promiscuously, and sometimes deposited on the external surface, radiating inwardly; they are, in fact, partially devitrified globules of glass, exactly similar to some artificial blow-pipe beads.

As is well known, glassy particles are sometimes given off from terrestrial volcanoes, but on entering the atmosphere they are immediately solidified and remain as mere fibres, like \emph{Pele's hair}, or as more or less irregular laminae, like pumice dust. The nearest approach to the globules in meteorites is met with in some artificial products. By directing a strong blast of hot air or steam into melted glassy furnace slag, it is blown into spray, and usually gives rise to pear-shaped globules, each having a long hair-like tail, which is formed because the surrounding air is too cold to retain the slag in a state of perfect fluidity. Very often the fibres are the chief product. I have never observed any such fibres in meteorites. If the slag be hot enough, some spheres are formed without tails, analogous to those characteristic of meteorites. The formation of such alone could not apparently occur unless the spray were blown into an atmosphere heated up to near the point of fusion, so that the glass might remain fluid until collected into globules. The retention of a true vitreous condition in such fused stony material would depend on both the chemical composition and the rate of cooling, and its permanent retention would in any case be impossible if the original glassy globule were afterwards kept for a long time at a temperature somewhat under that of fusion. The combination of all these conditions may very well be looked upon as unusual, and we may thus explain why grains containing true glass are comparatively very rare; but though rare they point out what was the origin of many others. In by far the greater number of cases the general basis has been completely devitrified, and the larger crystals are surrounded by a fine-grained stony mass. Other grains occur with a fan-shaped arrangement of crystalline needles, which an incautious, non-microscopical observer might confound with simple concretions. They have, however, a structure entirely different from any concretions met with in terrestrial rocks, as for example that of oolitic grains. In them we often see a well-marked nucleus, on which radiating crystals have been deposited equally on all sides, and the external form is manifestly due to the growth of these crystals. On the contrary the grains in meteorites now under consideration have an external form \emph{independent of the crystals}, which do not radiate from the centre, but from one or more places on the surface. They have, indeed, a structure absolutely identical with that of some artificial blowpipe beads which become crystalline on cooling. With a little care these can be made to crystallise from one point, and then the crystals shoot out from that point in a fan-shaped bundle, until the whole bead is altered. In this case we clearly see that the form of the bead was due to fusion, and existed prior to the formation of the crystals. The general structure of both these and the previously described spherical grains also shows that their rounded shape was not due to mechanical wearing. Moreover, melted globules with well-defined outline could not be formed in a mass of rock pressing against them on all sides, and I therefore argue that some at least of the constituent particles of meteorites were originally detached glassy globules, like drops of fiery rain.

Another remarkable character in the constituent particles of meteorites is that they are often mere fragments, although the entire body before being broken may originally have been only one-fortieth or one-fiftieth of an inch in diameter. It appears to me that thus to break such minute particles when they were probably in a separate state, mechanical forces of great intensity would be required. By far the greater number of meteorites have a structure which indicates that this breaking up of the constituents was of very general occurrence.

Assuming then that the particles were originally detached like volcanic ashes, it is quite clear that they were subsequently collected together and consolidated. This more than anything else appears to me a very great difficulty in the way of our adopting Reichenbach's cometary theory. Volcanic ashes are massed together and consolidated into tuff, because they are collected on the ground by the gravitative force of the earth. It appears to me very difficult to understand how in the case of a comet there could be in any part a sufficiently strong gravitative force to collect the dispersed dust into hard stony masses like meteorites. If it were not for this apparent difficulty we might suppose that some of the facts here described were due to the heat of the sun, when comets approach so near to it that the conditions may be practically almost solar. Comets may and probably do contain many meteorites, but I think that their structure indicates that they were originally formed under conditions far more like those now existing at the surface of the sun than in comets.

The particles having been collected together, the compound mass has evidently often undergone considerable mechanical and crystalline changes. The fragments have sometimes been broken \emph{in situ}, and "faulted;" and crystallisation has taken place, analogous to that met with in metamorphic rocks, which has more or less, and sometimes almost entirely, obliterated the original structure. The simplest explanation of this change is to suppose that after consolidation meteorites were variously heated to temperatures somewhat below their point of fusion. Those which have the structure of true lava may in some cases be portions which were actually remelted. We have also this striking fact, that meteoric masses of compound structure, themselves made up of fragments, have been again broken up into compound fragments, and these collected together and consolidated along with fresh material, to form the meteorites in their present condition. L'Aigle is a good example of this complex structure.

Another remarkable fact is the occurrence in some meteorites of many veins filled with material, in some respects so analogous to the black crust, that at one time I felt induced to believe that they were cracks, into which the crust had been injected. Akburfur is a good example of this, and seems to show that under whatever conditions the veins were found, they were injected not only with a black material, but also with iron and magnetic pyrites.

Taking, then, all the above facts into consideration, it appears to me that the conditions under which meteorites were formed must have been such that the temperature was high enough to fuse stony masses into glass; the particles could exist independently one of the other in an incandescent atmosphere, subject to violent mechanical disturbances; that the force of gravitation was great enough to collect these fine particles together into solid masses, and that these were in such a situation that they could be metamorphosed, further broken up into fragments, and again collected together. All these facts agree so admirably with what we know must now be taking place near the surface of the sun, that I cannot but think that, if we could only obtain specimens of the sun, we should find that their structure agreed very closely with that of meteorites. Considering also that the velocity with which the red flames have been seen to be thrown out from the sun is almost as great as that necessary to carry a solid body far out into planetary space, we cannot help wondering whether, after all, meteorites may not be portions of the sun recently detached from it by the violent disturbances which do most certainly now occur, or were carried off from it at some earlier period, when these disturbances were more intense. At the same time, as pointed out by me many years ago, some of the facts I have described may indicate that meteorites are the residual cosmical matter, not collected into planets, formed when the conditions now met with only near the surface of the sun extended much further out from the centre of the solar system. The chief objection to any great extension of this hypothesis is that we may doubt whether the force of gravitation would be sufficient to explain some of the facts. In any case I think that one or other of these solar theories, which to some extent agree with the speculations of the late Mr. Brailey, would explain the remarkable and very special microscopical structure of meteorites far better than that which refers them to portions of a volcanic planet, subsequently broken up, as advocated by Meunier, unless indeed we may venture to conclude that the material might still retain its original structure, due to very different conditions, previous to its becoming part of a planet. At the same time so little is positively known respecting the original constitution of the solar system, that all these conclusions must to some extent be looked upon as only provisional.

I will now proceed to consider some facts connected with meteoric irons. The so-called Widmanstatt's figuring, seen when some of these irons are acted on by acids, is well-known; but in my opinion the preparations are often very badly made. When properly prepared, the surface may be satisfactorily examined with a magnifying power of 200 linear, which is required to show the full detail. We may then see that the figuring is due to a very regular crystallisation, and to the separating out one from the other of different compounds of iron and nickel, and their phosphides. When meteoric iron showing this structure is artificially melted, the resulting product does not show the original structure, and it has therefore been contended that meteoric iron was never in a state of igneous fusion. In order to throw light on this question, I have paid very much attention to the microscopical structure of nearly all kinds of artificial irons and steels, by studying surfaces polished with very special care, so as to avoid any effect like burnishing, and then acting on them very carefully with extremely dilute nitric acid. In this manner most beautify and instructive specimens may be obtained, showing a very great amount of detail, and requiring a magnifying power varying up to at least 200 linear. In illustration of my subject I will call attention to only a few leading types of structure. In the first case we have grey pig-iron, showing laminae of graphite promiscuously arranged in all positions, on the surface of which is a thin layer of what is probably iron uncombined with carbon, whilst the intermediate spaces are filled up with what are probably two different compounds of iron and carbon.

White chilled refined iron has an entirely different structure and more uniform crystallisation, the structure is very remarkable and beautiful, mainly due to the varying crystallisation of an intensely hard compound of iron and carbon, and the two other softer compounds met with in grey pig.

Malleable bar iron has an entirely different structure, and shows fibres of black slag, and a more or less uniform crystallisation of iron with a varying small amount of carbon.

Cast steel differs again very much from any of the previous. It shows a fine-grained structure, due to small radiating crystals, and no plates of graphite.

The difference between any of the above and meteoric iron is extremely great.

In the case of Bessemer metal we have a crystalline structure approaching in some places more nearly to that of meteoric iron. We see a sort of Widmanstatt's figuring, but it is due to the separation of free iron from a compound containing a little carbon, and not to a variation in the amount of nickel.

The nearest approach to the structure of meteoric iron is met with in the central portion of thick bars of Swedish iron, kept for some weeks at a temperature below their melting point, but high enough to give rise to recrystallisation. We then get a complete separation of free iron from a compound containing some carbon, and a crystalline structure which, as far as mere form is concerned, most closely corresponds with that of meteoric iron, as may be at once seen on comparing them.

These facts clearly indicate that the Widmanstatt's figuring is the result of such a complete separation of the constituents and perfect crystallisation as can occur only when the process takes place slowly and gradually. They appear to me to show that meteoric iron was kept for a long time at a heat just below the point of fusion, and that we should be by no means justified in concluding that it was not previously melted. Similar principles are applicable in the case of the iron masses found in Disco, and it by no means follows that they are meteoric because they show the Widmanstatt's figuring. Difference in the rate of cooling would serve very well to explain the difference in the structure of some meteoric iron, which do not differ in chemical composition; but, as far as the general structure is concerned, I think that we are quite at liberty to conclude that all may have been melted, if this will better explain other phenomena. On this supposition we may account for the separation of the iron from the stony meteorites, since under conditions which brought into play only a moderate gravitative force, the melted iron would subside through the melted stone, as happens in our furnaces; whilst at the same time, as shown in my paper read at the meeting of the British Association in 1864, where the separating force of gravitation was small, they might remain mixed together, as in the Pallas iron, and others of that type.

In conclusion I would say that though from want of adequate material for investigation I feel that what I have so far done is very incomplete, yet I think that the facts I have described will, at all events, serve to prove that the method of study employed cannot fail to yield most valuable results, and to throw much light on many problems of great interest and importance in several different branches of science.
\clearpage
\end{document}
